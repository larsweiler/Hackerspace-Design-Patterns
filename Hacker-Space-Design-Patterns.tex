\documentclass[mathserif]{beamer}

\mode<presentation>
{
	\usetheme[compress]{Ilmenau}
	\useinnertheme{circles}
	\usecolortheme{c4}
	%\setbeamercovered{transparent}
	%\setbeamercovered{highly dynamic}
}


\usepackage[english]{babel}
\usepackage{ucs}
\usepackage[utf8x]{inputenc}
\PrerenderUnicode{ßüöäÜÖÄ„“…}
\usepackage[T1]{fontenc}

\newenvironment{itemizeframe}[1]
	{\begin{frame}{#1}\startitemizeframe}
	{\stopitemizeframe\end{frame}}
\newcommand\startitemizeframe{\begin{itemize}}
\newcommand\stopitemizeframe{\end{itemize}}


\title[A Hacker Space Design Pattern Catalogue]{\textbf{Building a Hacker Space}}

\author[J.~Ohlig, L.~Weiler]{%
	Jens Ohlig~\flq jens@ccc.de\frq \and Lars Weiler~\flq pylon@ccc.de\frq}

\institute[24C3]{24th Chaos Communication Congress}

\date[]{December 27, 2007}

\pgfdeclareimage[height=2.5cm]{labor}{pics/Labor-Panorama_(2006-09-15)}
\pgfdeclareimage[height=0.4cm]{CC-BY-NC-SA}{../CC-BY-NC-SA}
\titlegraphic{\pgfuseimage{labor}\\\href{http://creativecommons.org/licenses/by-nc-sa/2.0/de/}{\pgfuseimage{CC-BY-NC-SA}}}

\pgfdeclareimage[height=0.8cm]{pesthoernchen}{../Pesthoernchen}
\logo{\pgfuseimage{pesthoernchen}}


\begin{document}

\begin{frame}
  \titlepage
\end{frame}

\AtBeginSubsection[]
{
	\begin{frame}{Outline}
		\tableofcontents[currentsection,currentsubsection]
	\end{frame}
}

\begin{frame}{Outline}
  \tableofcontents[hideallsubsections]
  % Die Option [pausesections] könnte nützlich sein.
\end{frame}

\section{Introduction}

\subsection{Who we are}

\begin{frame}{The speakers}
	\begin{columns}
		\column{.5\textwidth}
		\center{\underline{Jens}}
		\begin{itemize}
			\item Co-Founder of Chaos Computer Club Cologne (C4)
			\item Still active there
			\item CCC-activist for more than 15 years
			\item CCC-spokesman, board-member etc.
		\end{itemize}
		\column{.5\textwidth}
		\center{\underline{Pylon}}
		\begin{itemize}
			\item Co-Founder of Chaos Computer Club Düsseldorf (Chaosdorf)
			\item Now active in Cologne
			\item CCC-activist for more than eight years
			\item CCC-spokesman, board-member etc.
		\end{itemize}
	\end{columns}
\end{frame}

\begin{itemizeframe}{Chaos Computer Club Cologne}
	\item Founded in 1997
	\item around 42 members
	\item currently in Version 3.5 of our hacker space, operational in that
		location since 1999
	\item All pictures in this presentation has been taken in the C4 hacker space
\end{itemizeframe}

% Warum?

\subsection{Why this catalogue?}

\begin{itemizeframe}{Thanks to the August 2007 hacker space tour}
	\item A group of American hackers visited hacker spaces in Germany and Austria
	\item They wanted to know how our European hacker spaces work
	\item After the Camp they visited a couple of hacker spaces
	\item Every hacker space did a presentation about their history
	\item We created some Design Patterns
\end{itemizeframe}

\begin{itemizeframe}{Design Patterns}
	\item Historically used for urban planning
	\item Transfered for typical situations in software development
	\item Problem $\rightarrow$ Implementation
\end{itemizeframe}

\begin{itemizeframe}{What we want to tell you}
	\item We want to share our knowledge of building our own hacker space
	\item We won't give you a detailed manual
	\item Your mileage may vary
\end{itemizeframe}

% Design-Patterns

\section[Design Patterns]{The Hacker Space Design Patterns Catalogue}

\subsection{Sustainability Patterns}

\begin{frame}{The Infrastructure Pattern}
	\begin{alertblock}{Problem}
		You have a chicken-and-egg-problem: What should come first?  Infrastructure
		or projects?
	\end{alertblock}
	\pause
	\begin{exampleblock}{Implementation}
		Make everything \textbf{infrastructure-driven}.  Rooms, power, servers,
		connectivity, and other facilities come first.  Once you have that, people
		will come up with the most amazing projects you didn't think about in the
		first place.
	\end{exampleblock}
\end{frame}

\pgfdeclareimage[width=\textwidth]{noc}{pics/crw_0713_round}
\begin{frame}[plain]{NOC with Server Racks}
	\begin{center}
		\pgfuseimage{noc}
	\end{center}
\end{frame}

\begin{frame}{The Grace Hopper Pattern}
	\begin{alertblock}{Problem}
		Is now really the time to start your hacker space?  Shouldn't you wait?
		Have you really thought of all the problems?
	\end{alertblock}
	\pause
	\begin{exampleblock}{Implementation}
		\textbf{Sure it is the time!}
		\begin{quote}
			It's always easier to ask forgiveness than it is to get permission.
		\end{quote}
		(Grace Hopper, US Navy Rear Admiral and computer scientist)

		It's important to start.  Many problems you think of before will vanish as
		soon as you get started.  \textbf{When in doubt, do it!}
	\end{exampleblock}
\end{frame}

\begin{frame}{The Community Pattern}
	\begin{alertblock}{Problem}
		How should your group communicate?
	\end{alertblock}
	\pause
	\begin{exampleblock}{Implementation}
		You are hackers, you know what to do.  Stop slacking and set up a mailing
		list, a wiki, and an IRC channel.  You will need all three.  Think about a
		\textbf{platform for discussion}, storage for \textbf{documentation}
		and \textbf{real-time communication}.
	\end{exampleblock}
\end{frame}

\begin{frame}{The Critical Mass Pattern}
	\begin{alertblock}{Problem}
		You want to set up a hacker space in your city alone.  You fail.
	\end{alertblock}
	\pause
	\begin{exampleblock}{Implementation}
		The rule of thumb is \textbf{2\,+\,2}.  You need a partner to get the initial
		idea kicked off, making two of you.  You need two more people in order to
		get real work done.  Don't start before you are at least four people.  From
		this point it's easy to recruit more people.  \textbf{Aim for ten people} for
		a start.
	\end{exampleblock}
\end{frame}

\begin{frame}{The Strong Personalities Pattern}
	\begin{alertblock}{Problem}
		Nothing gets done.  You all want the hacker space, but it's so hard to get
		off your asses.
	\end{alertblock}
	\pause
	\begin{exampleblock}{Implementation}
		Look for \textbf{strong personalities} as members of your original group.
		You will need people with \textbf{experience in building structures}.  Look
		for people who \emph{have} authority (and get respect), not for people who
		\emph{use} authority (and get laughed at).
	\end{exampleblock}
\end{frame}


\subsection{Independence Patterns}

\begin{frame}{The Landlord and Neighbourhood Pattern}
	\begin{alertblock}{Problem}
		You have found the perfect hacker space, but the landlord seems to be weird.
		Also, the neighbours are picky.
	\end{alertblock}
	\pause
	\begin{exampleblock}{Implementation}
		Choose wisely.  A benevolent, but \textbf{uninterested landlord} and
		\textbf{cool neighbours} can be the decisive reasons why the hacker space
		takes off or not.  Not so cool neighbours may call the cops at 2~AM.
		Depending on your projects, this may be a serious problem.  As hackers you
		do not live the majority lifestyle---look for neighbours who are also weird
		and outside the majority.
	\end{exampleblock}
\end{frame}

\begin{frame}{The Roommate Anti-Pattern}
	\begin{alertblock}{Problem}
		You need a space for meetings and as a lab, to store and work on materials
		for projects.  In order to minimize rent or out of sympathy, you think it's
		great when someone lives in your space. But somehow it doesn't work, as you
		cannot use the lab anymore.
	\end{alertblock}
	\pause
	\begin{exampleblock}{Implementation}
		Guest are fine, but \textbf{don't let anyone live there}.  Kick them out if
		necessary.
	\end{exampleblock}
\end{frame}

\pgfdeclareimage[height=1.5cm]{plenarsaal}{pics/Plenarsaal-Panorama_(2006-09-26)_round}
\begin{frame}{The Séparée Pattern}
	\begin{alertblock}{Problem}
		You want to chill, discuss, or work in small groups.  But the main room is
		occupied: There are simply too many people at your space.  Or you want to
		smoke a cigarette at the space without disturbing non-smokers.
	\end{alertblock}
	\pause
	\begin{exampleblock}{Implementation}
		Look for a hacker space with \textbf{smaller, separate rooms}.  Use
		\textbf{curtains} or \textbf{doors} to separate them from the main room.
		Separate rooms can also be used for smokers in a non-smoking hacker space.
	\end{exampleblock}
	\begin{center}
		\pgfuseimage{plenarsaal}
	\end{center}
\end{frame}

\begin{frame}{The Kitchen Pattern}
	\begin{alertblock}{Problem}
		As a human being, you need food.  As a hacker, you need caffeine and food at
		odd times.
	\end{alertblock}
	\pause
	\begin{exampleblock}{Implementation}
		Have a \textbf{kitchen} at your space.  Nothing brings people together like
		\textbf{cooking} together.  Have \textbf{fridges} for Club-Mate.  Selling
		soft-drinks will help you raise money for the rent.  Invest in the single
		most important piece of hardware: a \textbf{dishwasher}.  Have a
		\textbf{freezer} for pizzas and buy decent \textbf{kitchen equipment}.  Show
		nerds how to cook \textbf{real food}.
	\end{exampleblock}
\end{frame}

\pgfdeclareimage[width=\textwidth]{kitchen}{pics/crw_0716_round}
\begin{frame}[plain]{The Kitchen}
	\begin{center}
		\pgfuseimage{kitchen}
	\end{center}
\end{frame}

\begin{frame}{The Coziness Pattern}
	\begin{alertblock}{Problem}
		All work and no play makes Jack a dull boy.  There must be something else
		than only workstations and electronics.
	\end{alertblock}
	\pause
	\begin{exampleblock}{Implementation}
		Bring in \textbf{couches, sofas, comfortable chairs, tables, ashtrays,
		ambient light,} \textbf{stereo equipment, a projector, and video game consoles}.
		Bringing in plants didn't work for us.
	\end{exampleblock}
\end{frame}

\pgfdeclareimage[width=\textwidth]{fnordcenter}{pics/crw_0712_round}
\begin{frame}[plain]{The Fnordcenter}
	\begin{center}
		\pgfuseimage{fnordcenter}
	\end{center}
\end{frame}

\pgfdeclareimage[height=7cm]{leaves}{pics/plant-leaves_round}
\pgfdeclareimage[height=7cm]{no-leaves}{pics/plant-no-leaves_round}
\begin{frame}[plain]{Our Poor Plant Named “Egor”…}
	\begin{columns}
		\column{.5\textwidth}
			\begin{center}
				\pgfuseimage{leaves}\\
				Sep. '06
			\end{center}
		\column{.5\textwidth}
			\begin{center}
				\pgfuseimage{no-leaves}\\
				Dec. '07
			\end{center}
	\end{columns}
\end{frame}


\begin{frame}{The Shower Pattern}
	\begin{alertblock}{Problem}
		After long hacking sessions, you will start to smell funny.  Also, guests to
		your space seem to neglect personal hygiene.
	\end{alertblock}
	\pause
	\begin{exampleblock}{Implementation}
		The discriminate hacker space has a \textbf{bathroom with a shower}.  After a
		long hacking night you'll have the best ideas while taking a shower.  Guests
		from other hacker spaces may stay for several days.  Ideally you will buy a
		\textbf{washing machine} to get rid of all the smelly towels.
	\end{exampleblock}
\end{frame}

\pgfdeclareimage[width=\textwidth]{shower}{pics/crw_0718_round}
\begin{frame}[plain]{The Bathroom (extract)}
	\begin{center}
		\pgfuseimage{shower}
	\end{center}
\end{frame}

\begin{frame}{The Membership Fees Pattern}
	\begin{alertblock}{Problem}
		You need to pay your rent and utilities.  Larger projects need to be funded.
	\end{alertblock}
	\pause
	\begin{exampleblock}{Implementation}
		\textbf{Collect fees regularly}.  Make no exceptions, ever.  Choose an
		appropriate amount.  Have discounts for students.  Have at least three
		months of rent on your account, all the time, no exceptions.  Elect a
		totalitarian treasurer.
	\end{exampleblock}
\end{frame}

\begin{frame}{The Sponsoring Anti-Pattern}
	\begin{alertblock}{Problem}
		You think it's a good idea to meet at a company that likes you or at a
		university where most of you study anyway.
	\end{alertblock}
	\pause
	\begin{exampleblock}{Implementation}
		\textbf{Never ever depend} your space \textbf{on external sponsors}.
		Donations are great, but remember that companies can go bankrupt and you
		won't be a student forever.  Meeting at a university will exclude
		high-school kids or people who don't like the university culture.  No
		company, no matter how nice, will give away presents forever without asking
		for favours in return.  That's capitalism…
	\end{exampleblock}
\end{frame}


\subsection{Regularity Patterns}

\begin{frame}{The Plenum Pattern}
	\begin{alertblock}{Problem}
		You want to resolve internal conflicts, exercise democratic
		decision-making, and discuss recent issues and future plans.
	\end{alertblock}
	\pause
	\begin{exampleblock}{Implementation}
		Have a \textbf{regular meeting} with possibly all members.  Have an
		\textbf{agenda} and \textbf{set goals}.  Make people commit themselves to
		tasks.  Write down \textbf{minutes of the meeting} and post them on a mailing
		list and/or Wiki.  Go for the only date that works: \textbf{once a week}.
		Weird dates like “first full-moon after the third Friday” will never work.
		Likewise doesn't every other week or anything similar.
	\end{exampleblock}
\end{frame}

\begin{frame}{The Tuesday Pattern}
	\begin{alertblock}{Problem}
		Every weekday sucks.  You will not find any day when every hacker can attend
		a meeting.  Someone always has an appointment.
	\end{alertblock}
	\pause
	\begin{exampleblock}{Implementation}
		\textbf{Meet on Tuesday}.  Since all days are equally bad, just pick the
		Tuesday.  End of discussion.
	\end{exampleblock}
\end{frame}

\begin{frame}{The OpenChaos Pattern}
	\begin{alertblock}{Problem}
		You want to draw in new people and provide an interface to the outside
		world.
	\end{alertblock}
	\pause
	\begin{exampleblock}{Implementation}
		Have a \textbf{monthly, public, and open lecture, talk or workshop}.
		Announce it at your local time (no UTC, CEST, EST or something else).
		Invite interesting visitors to your regular meetings and don't tell the
		weirdos about them.	\end{exampleblock}
\end{frame}

\begin{frame}{The U23 Pattern}
	\begin{alertblock}{Problem}
		Your older members graduate from college or get married.  Your space needs
		fresh blood.
	\end{alertblock}
	\pause
	\begin{exampleblock}{Implementation}
		Recruit young people through a \textbf{challenge} you set up for them, in
		form of a \textbf{course that spans several weeks}.  Overwhelm them with
		problems from hardware and software hacking and let them solve it in teams.
		Prepare for the challenge and \textbf{tutor them}, but give them room to
		experiment.  Retire after the team-building and let the smartest of the
		young ones run the space.
	\end{exampleblock}
\end{frame}

\begin{frame}{The Sine Curve Pattern}
	\begin{alertblock}{Problem}
		You did everything right.  You had some big events and a nice time in your
		shiny hacker space.  But after some time the enthusiasm goes away and your
		projects are stagnating.
	\end{alertblock}
	\pause
	\begin{exampleblock}{Implementation}
		Peak enthusiasm at a hacker space has the form of a \textbf{sine curve with
		a cycle duration of four years}.  Keep the hacker space running, even if the
		feel-good-factor is temporarly on holidays.  Chances are your space will be
		awesome again in two years.  \textbf{Don't give up!}  Maybe an exciting new
		member will knock on your door tomorrow.
	\end{exampleblock}
\end{frame}


\subsection{Conflict Resolution Patterns}

\begin{frame}{The Consensus Pattern}
	\begin{alertblock}{Problem}
		You need a group decision and want to make sure no one gets left behind.
	\end{alertblock}
	\pause
	\begin{exampleblock}{Implementation}
		Use the weekly plenum for discussion. Don't take votes---\textbf{discuss
		until everyone agrees}.

		For some problems this pattern is the best.
	\end{exampleblock}
\end{frame}

\begin{frame}{The Democracy Pattern}
	\begin{alertblock}{Problem}
		You need to make a group decision.  Discussion does not seem to lead you
		anywhere.
	\end{alertblock}
	\pause
	\begin{exampleblock}{Implementation}
		Use the weekly plenum for discussion.  Do take votes---\textbf{the strongest
		minority wins over the weaker minorities}.

		For some problems this pattern is the best.
	\end{exampleblock}
\end{frame}

\begin{frame}{The Command Pattern}
	\begin{alertblock}{Problem}
		Nobody does the dishes.  Your hacker space looks crappy.  No one seems to
		care.
	\end{alertblock}
	\pause
	\begin{exampleblock}{Implementation}
		\textbf{Order people} to do the dishes, take out the trash, keep the
		infrastructure up and running.  \textbf{Yell, if necessary!  But always participate}.

		For some problems this pattern is the best.
	\end{exampleblock}
\end{frame}

\begin{frame}{The \texttt{sudo leadership} Pattern}
	\begin{alertblock}{Problem}
		You started as a community of like-minded people, but suddenly you find
		yourself in a dictatorship run by a single hacker.
	\end{alertblock}
	\pause
	\begin{exampleblock}{Implementation}
		Do not have ranks.  \textbf{Use leadership temporarily}, like for projects
		and when you really need it.  \textbf{Don't have a single root}.
	\end{exampleblock}
\end{frame}

\begin{frame}{The Responsibility Pattern}
	\begin{alertblock}{Problem}
		You volunteered for the task of running a critical piece of infrastructure,
		e.g. the mail server, but you feel the sudden urge to slack.
	\end{alertblock}
	\pause
	\begin{exampleblock}{Implementation}
		Just because volunteer work doesn't get paid doesn't mean it's less
		important.  Remember that you will directly hurt your friends and the hacker
		space.  \textbf{Take pride in your volunteer work}.  It will make you grow
		stronger as a person and is satisfying.  When you realise that you really
		cannot do the job any more, \textbf{your last task is to hand it over}.
	\end{exampleblock}
\end{frame}

\begin{frame}{The Debate Culture Pattern}
	\begin{alertblock}{Problem}
		You are in the middle of your weekly plenum.  Everybody's yelling, nothing
		gets done.
	\end{alertblock}
	\pause
	\begin{exampleblock}{Implementation}
		Many geeks have very poor debate skills, the result of years of flame wars
		on the Net.  Make \textbf{people with actual social skills lead the
		discussion}.  Those with a background in real-life political work (e.g.
		student council) were best for our group.  \textbf{Learn from them}.
		\textbf{Learn not to interrupt others}.
	\end{exampleblock}
\end{frame}

\begin{frame}{The Bikeshed Anti-Pattern}
	\begin{alertblock}{Problem}
		You suggest creating something new for your hacker space, like a bikeshed.
		But now everybody discuss about it's colour.  No bikeshed will be built.
	\end{alertblock}
	\pause
	\begin{exampleblock}{Implementation}
		That's a known problem.  If you suggest something what everybody else in
		your hacker space can build, \textbf{they will take part in the
		discussion}.  And if it's only the colour of the bikeshed, the design of the
		T-shirts, the Linux-distribution on the server, etc.  Nerds tend to discuss
		trivial problems in epic detail, while more complex tasks will be ignored.
		\textbf{Identify pointless discussion} like these and just end them.
	\end{exampleblock}
\end{frame}

\begin{frame}[plain]{History of the Bikeshed-Problem}
	\begin{quote}
		C. Northcote Parkinson wrote a book in the early 1960s, called “Parkinson's
		Law”, which contains a lot of insight into the dynamics of management.\\
		\[…\]\\
		In the specific example involving the bike shed, the other vital component
		is an atomic power-plant, I guess that illustrates the age of the book.
		Parkinson shows how you can go into the board of directors and get approval
		for building a multi-million or even billion dollar atomic power plant, but
		if you want to build a bike shed you will be tangled up in endless
		discussions.
	\end{quote}

	see \url{http://www.bikeshed.com/}
\end{frame}

\begin{frame}{The Private Talk Pattern}
	\begin{alertblock}{Problem}
		Someone causes a problem that cannot be resolved in the group.
	\end{alertblock}
	\pause
	\begin{exampleblock}{Implementation}
		Let some experienced member of your group \textbf{talk} to the trouble-maker
		\textbf{in private}.  \textbf{Listen to the person}.  Let them know how the
		group feels about the problem without exposing them in front of the group.
	\end{exampleblock}
\end{frame}


\subsection{Creative Chaos Patterns}

\begin{frame}{The Old Hardware Pattern}
	\begin{alertblock}{Problem}
		You can't bring in shiny new hardware, as there is no space left.  Your
		space has become a hardware museum filled with junk.
	\end{alertblock}
	\pause
	\begin{exampleblock}{Implementation}
		Create a \textbf{pile/stack} where you put that old, unused hardware on.
		Let everybody take from it.  Anything left within a while \textbf{should
		be thrown away}.  But make sure you \textbf{announce that step} not only
		once, but at least three times with an escalation system.
	\end{exampleblock}
\end{frame}

\begin{frame}[plain]{(Sorry, in German only)}
	\begin{quote}
		\textbf{Bereitstellung von Hardware im Chaoslabor}

		§1 Das Chaoslabor ist ein Bereich vorbildlicher Ordnung und Sauberkeit, in
		dem der Chaos Computer Club Cologne e.V. seine Vereinstätigkeit ausübt.

		§2 Unter Aufräumpersonal werden Personen verstanden, die sich um den
		Zustand des Clubraumes kümmern. Aufräumpersonal genießt Heldenstatus und
		Immunität gegenüber Anfeindungen von Besitzern nicht funktionierender
		Hardware.

		\[…\]\\
	\end{quote}
	see \url{http://wiki.koeln.ccc.de/index.php?title=Hacker_Space/Hardware-Gesetz}
\end{frame}

\pgfdeclareimage[width=\textwidth]{stack}{pics/crw_0723_round}
\begin{frame}[plain]{A Stack of Old Hardware}
	\begin{center}
		\pgfuseimage{stack}
	\end{center}
\end{frame}

\begin{frame}{The Key Pattern}
	\begin{alertblock}{Problem}
		You want the hacker space accessable all the time.  You don not want to call
		somebody else during night to lock the hacker space when you leave.
	\end{alertblock}
	\pause
	\begin{exampleblock}{Implementation}
		\textbf{Hand out keys}.  Track who owns a key.  \textbf{Have a good
		lock} so that nobody can copy the key without your permission.  Collect a
		deposit for the key, so that the owner takes care for it.  Or build a nice
		electronic locking system (with all cool things and all messy problems)…
	\end{exampleblock}
\end{frame}

\begin{frame}{The Club Mate Pattern}
	\begin{alertblock}{Problem}
		You need to raise funds.  You want to stay up longer during night.  You want
		to receive really good impressions without drugs.
	\end{alertblock}
	\pause
	\begin{exampleblock}{Implementation}
		Buy at least one pallet of \textbf{Club-Mate} and sell it in your hacker
		space.  You will realise the results very soon!
	\end{exampleblock}
\end{frame}

\pgfdeclareimage[width=\textwidth]{club-mate}{pics/crw_0727_round}
\begin{frame}[plain]{Golden Club-Mate}
	\begin{center}
		\pgfuseimage{club-mate}
	\end{center}
\end{frame}


\section{Conclusion}

\subsection{This is not a cookbook}
\begin{itemizeframe}{Conclusion}
	\item There is no “golden way” building up a hacker space
	\item Based on experience there are a couple of patterns which might match
	\item Be creative!  Try out your own way!
		\vspace{2em}
	\item Question \& Answer session @24C3: Day 1, 17:15, Workshoproom A
\end{itemizeframe}

\subsection{Find your nearest Hacker Space}
\begin{frame}{Hacker Spaces (not complete)}
	Germany, Austria, Switzerland:
	\begin{itemize}
		\item CCC-based $\rightarrow$ \url{http://www.ccc.de/regional/}
		\item Netzladen $\rightarrow$ \url{http;//www.netzladen.org/}
		\item Das Labor $\rightarrow$ \url{http://www.das-labor.org/}
		\item c-base $\rightarrow$ \url{http://www.c-base.org/}
	\end{itemize}
	USA:
	\begin{itemize}
		\item NYC Resistor $\rightarrow$ \url{http://www.nycresistor.com/}
		\item Seattle
		\item San Francisco
		\item L.A.
	\end{itemize}
	Australia:
	\begin{itemize}
		\item TheHacktory $\rightarrow$ \url{http://thehacktory.com/}
	\end{itemize}
\end{frame}

\end{document}


